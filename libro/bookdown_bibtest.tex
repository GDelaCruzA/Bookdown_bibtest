% Options for packages loaded elsewhere
\PassOptionsToPackage{unicode}{hyperref}
\PassOptionsToPackage{hyphens}{url}
%
\documentclass[
  12pt,
  twoside]{book}
\usepackage{amsmath,amssymb}
\usepackage{lmodern}
\usepackage{iftex}
\ifPDFTeX
  \usepackage[T1]{fontenc}
  \usepackage[utf8]{inputenc}
  \usepackage{textcomp} % provide euro and other symbols
\else % if luatex or xetex
  \usepackage{unicode-math}
  \defaultfontfeatures{Scale=MatchLowercase}
  \defaultfontfeatures[\rmfamily]{Ligatures=TeX,Scale=1}
\fi
% Use upquote if available, for straight quotes in verbatim environments
\IfFileExists{upquote.sty}{\usepackage{upquote}}{}
\IfFileExists{microtype.sty}{% use microtype if available
  \usepackage[]{microtype}
  \UseMicrotypeSet[protrusion]{basicmath} % disable protrusion for tt fonts
}{}
\usepackage{xcolor}
\IfFileExists{xurl.sty}{\usepackage{xurl}}{} % add URL line breaks if available
\IfFileExists{bookmark.sty}{\usepackage{bookmark}}{\usepackage{hyperref}}
\hypersetup{
  pdftitle={A Minimal Book Example},
  pdfauthor={John Doe},
  hidelinks,
  pdfcreator={LaTeX via pandoc}}
\urlstyle{same} % disable monospaced font for URLs
\usepackage{color}
\usepackage{fancyvrb}
\newcommand{\VerbBar}{|}
\newcommand{\VERB}{\Verb[commandchars=\\\{\}]}
\DefineVerbatimEnvironment{Highlighting}{Verbatim}{commandchars=\\\{\}}
% Add ',fontsize=\small' for more characters per line
\usepackage{framed}
\definecolor{shadecolor}{RGB}{248,248,248}
\newenvironment{Shaded}{\begin{snugshade}}{\end{snugshade}}
\newcommand{\AlertTok}[1]{\textcolor[rgb]{0.94,0.16,0.16}{#1}}
\newcommand{\AnnotationTok}[1]{\textcolor[rgb]{0.56,0.35,0.01}{\textbf{\textit{#1}}}}
\newcommand{\AttributeTok}[1]{\textcolor[rgb]{0.77,0.63,0.00}{#1}}
\newcommand{\BaseNTok}[1]{\textcolor[rgb]{0.00,0.00,0.81}{#1}}
\newcommand{\BuiltInTok}[1]{#1}
\newcommand{\CharTok}[1]{\textcolor[rgb]{0.31,0.60,0.02}{#1}}
\newcommand{\CommentTok}[1]{\textcolor[rgb]{0.56,0.35,0.01}{\textit{#1}}}
\newcommand{\CommentVarTok}[1]{\textcolor[rgb]{0.56,0.35,0.01}{\textbf{\textit{#1}}}}
\newcommand{\ConstantTok}[1]{\textcolor[rgb]{0.00,0.00,0.00}{#1}}
\newcommand{\ControlFlowTok}[1]{\textcolor[rgb]{0.13,0.29,0.53}{\textbf{#1}}}
\newcommand{\DataTypeTok}[1]{\textcolor[rgb]{0.13,0.29,0.53}{#1}}
\newcommand{\DecValTok}[1]{\textcolor[rgb]{0.00,0.00,0.81}{#1}}
\newcommand{\DocumentationTok}[1]{\textcolor[rgb]{0.56,0.35,0.01}{\textbf{\textit{#1}}}}
\newcommand{\ErrorTok}[1]{\textcolor[rgb]{0.64,0.00,0.00}{\textbf{#1}}}
\newcommand{\ExtensionTok}[1]{#1}
\newcommand{\FloatTok}[1]{\textcolor[rgb]{0.00,0.00,0.81}{#1}}
\newcommand{\FunctionTok}[1]{\textcolor[rgb]{0.00,0.00,0.00}{#1}}
\newcommand{\ImportTok}[1]{#1}
\newcommand{\InformationTok}[1]{\textcolor[rgb]{0.56,0.35,0.01}{\textbf{\textit{#1}}}}
\newcommand{\KeywordTok}[1]{\textcolor[rgb]{0.13,0.29,0.53}{\textbf{#1}}}
\newcommand{\NormalTok}[1]{#1}
\newcommand{\OperatorTok}[1]{\textcolor[rgb]{0.81,0.36,0.00}{\textbf{#1}}}
\newcommand{\OtherTok}[1]{\textcolor[rgb]{0.56,0.35,0.01}{#1}}
\newcommand{\PreprocessorTok}[1]{\textcolor[rgb]{0.56,0.35,0.01}{\textit{#1}}}
\newcommand{\RegionMarkerTok}[1]{#1}
\newcommand{\SpecialCharTok}[1]{\textcolor[rgb]{0.00,0.00,0.00}{#1}}
\newcommand{\SpecialStringTok}[1]{\textcolor[rgb]{0.31,0.60,0.02}{#1}}
\newcommand{\StringTok}[1]{\textcolor[rgb]{0.31,0.60,0.02}{#1}}
\newcommand{\VariableTok}[1]{\textcolor[rgb]{0.00,0.00,0.00}{#1}}
\newcommand{\VerbatimStringTok}[1]{\textcolor[rgb]{0.31,0.60,0.02}{#1}}
\newcommand{\WarningTok}[1]{\textcolor[rgb]{0.56,0.35,0.01}{\textbf{\textit{#1}}}}
\usepackage{longtable,booktabs,array}
\usepackage{calc} % for calculating minipage widths
% Correct order of tables after \paragraph or \subparagraph
\usepackage{etoolbox}
\makeatletter
\patchcmd\longtable{\par}{\if@noskipsec\mbox{}\fi\par}{}{}
\makeatother
% Allow footnotes in longtable head/foot
\IfFileExists{footnotehyper.sty}{\usepackage{footnotehyper}}{\usepackage{footnote}}
\makesavenoteenv{longtable}
\usepackage{graphicx}
\makeatletter
\def\maxwidth{\ifdim\Gin@nat@width>\linewidth\linewidth\else\Gin@nat@width\fi}
\def\maxheight{\ifdim\Gin@nat@height>\textheight\textheight\else\Gin@nat@height\fi}
\makeatother
% Scale images if necessary, so that they will not overflow the page
% margins by default, and it is still possible to overwrite the defaults
% using explicit options in \includegraphics[width, height, ...]{}
\setkeys{Gin}{width=\maxwidth,height=\maxheight,keepaspectratio}
% Set default figure placement to htbp
\makeatletter
\def\fps@figure{htbp}
\makeatother
\setlength{\emergencystretch}{3em} % prevent overfull lines
\providecommand{\tightlist}{%
  \setlength{\itemsep}{0pt}\setlength{\parskip}{0pt}}
\setcounter{secnumdepth}{-\maxdimen} % remove section numbering
\newlength{\cslhangindent}
\setlength{\cslhangindent}{1.5em}
\newlength{\csllabelwidth}
\setlength{\csllabelwidth}{3em}
\newlength{\cslentryspacingunit} % times entry-spacing
\setlength{\cslentryspacingunit}{\parskip}
\newenvironment{CSLReferences}[2] % #1 hanging-ident, #2 entry spacing
 {% don't indent paragraphs
  \setlength{\parindent}{0pt}
  % turn on hanging indent if param 1 is 1
  \ifodd #1
  \let\oldpar\par
  \def\par{\hangindent=\cslhangindent\oldpar}
  \fi
  % set entry spacing
  \setlength{\parskip}{#2\cslentryspacingunit}
 }%
 {}
\usepackage{calc}
\newcommand{\CSLBlock}[1]{#1\hfill\break}
\newcommand{\CSLLeftMargin}[1]{\parbox[t]{\csllabelwidth}{#1}}
\newcommand{\CSLRightInline}[1]{\parbox[t]{\linewidth - \csllabelwidth}{#1}\break}
\newcommand{\CSLIndent}[1]{\hspace{\cslhangindent}#1}
% acumulado de diversas fuentes... la mayoría perdidas, sorry
% única línea en el ejemplo \usepackage{booktabs}
%linea para evitar error de bookdown 
%You can't use `macro parameter character #` in vertical mode
%  otra opcion es \usepackage{lscape}
%\usepackage{pdflscape} comentado por el header-include en index.rmd
% seccion 6.1 del manual de bookdown latex-preamble.html
%para cambiar el font en stackoverflow
\usepackage{fontspec}
%\setmainfont{Arial}
%bloquemos el Sans
%\setmainfont{Liberation Sans}
%
\usepackage{booktabs}
%para regular una línea después de cada párrafo:
%https://tex.stackexchange.com/questions/135134/how-to-add-an-empty-line-between-paragraphs
\usepackage{parskip}
% soporte para tablas largas; bookdown text
\usepackage{longtable}
% se agrega longtable = TRUE a knitr::kable
% de Berry how to Thesis
\usepackage[none]{hyphenat}
\pagestyle{plain}
\raggedbottom 
%Writing a book with bookdown in foreign languages _
%Anh Hoang Duc (2020-09-26 5_35_46 PM).html
%\usepackage[utf8]{inputenc} no se usa: ignored with utf8 based engines
% \usepackage[spanish]{babel} error de redundancia
%\usepackage[spanish, es-tabla]{babel}
%bookdown_intro_sp.pdf Fernández-Casal y Cotos-Yáñez, 2018
\ifxetex
  \usepackage{polyglossia}
  \setmainlanguage{spanish}
 % Tabla en lugar de cuadro
  \gappto\captionsspanish{\renewcommand{\tablename}{Tabla}
          \renewcommand{\listtablename}{Lista de tablas}}
          
\else
  \usepackage[spanish,es-tabla]{babel}
\fi
% cambiar los títulos de las listas
% https://es.overleaf.com/learn/latex/Lists_of_tables_and_figures
% https://latexref.xyz/Table-of-contents-etc_002e.html
\addto\captionsspanish{\renewcommand{\listfigurename}{Lista de figuras}}
\addto\captionsspanish{\renewcommand{\contentsname}{Índice de contenido}}
%\renewcommand{\listfigurename}{Lista de figuras}
%\renewcommand{\listtablename}{Lista de tablas}
% define la paginación con numerales romanos hasta antes del primer capítulo
\frontmatter
%para definir el margen del escrito para todos sus lados
\usepackage[margin=1.0in]{geometry}
% solucion a bibliografia en el í5ndice en pdf en libro
% por ahora, comentar para el Bookdown_bibtest"   "
% https://github.com/rstudio/bookdown/issues/192
%\usepackage{makeidx}
% para indexar apéndices y listas de figs y tablas
% https://stackoverflow.com/questions/56637936/how-to-add-list-of-figures-and-list-of-tables-to-the-table-of-content-with-r-mar
\usepackage[nottoc]{tocbibind}
%\usepackage{tocbibind}
% para que se procese en índice de contenido, agregar:
\usepackage{makeidx}
\makeindex
% Then insert \printindex at the end of your book through the YAML option includes -> after_body.
% 
\ifLuaTeX
  \usepackage{selnolig}  % disable illegal ligatures
\fi

\title{A Minimal Book Example}
\author{John Doe}
\date{2022-07-22}

\usepackage{amsthm}
\newtheorem{theorem}{Teorema}
\newtheorem{lemma}{Lema}
\newtheorem{corollary}{Corolario}
\newtheorem{proposition}{Proposición}
\newtheorem{conjecture}{Conjecture}
\theoremstyle{definition}
\newtheorem{definition}{Definición}
\theoremstyle{definition}
\newtheorem{example}{Ejemplo}
\theoremstyle{definition}
\newtheorem{exercise}{Ejercicio}
\theoremstyle{definition}
\newtheorem{hypothesis}{Hypothesis}
\theoremstyle{remark}
\newtheorem*{remark}{Nota: }
\newtheorem*{solution}{Solución}
\begin{document}
\maketitle

\hypertarget{acta-de-revisiuxf3n}{%
\chapter*{Acta de revisión}\label{acta-de-revisiuxf3n}}
\addcontentsline{toc}{chapter}{Acta de revisión}

título del capítulo en blanco\ldots{}

Los subtítulos de segundo nivel sin numeración en el índice

\hypertarget{carta-de-sesiuxf3n-de-derechos}{%
\chapter*{Carta de Sesión de Derechos}\label{carta-de-sesiuxf3n-de-derechos}}
\addcontentsline{toc}{chapter}{Carta de Sesión de Derechos}

Aquí va la carta.

\newpage

\hypertarget{dedicatoria}{%
\chapter*{Dedicatoria}\label{dedicatoria}}
\addcontentsline{toc}{chapter}{Dedicatoria}

Generalmente una página con la dedicatoria

\newpage

\hypertarget{agradecimientos}{%
\chapter*{Agradecimientos}\label{agradecimientos}}
\addcontentsline{toc}{chapter}{Agradecimientos}

Todo el rollo de agradecimiento a personas e instituciones, incluyendo al comité.
\url{https://community.rstudio.com/t/bookdown-adding-abstract-before-the-table-of-contents/47149/5}

\url{https://stackoverflow.com/questions/56637936/how-to-add-list-of-figures-and-list-of-tables-to-the-table-of-content-with-r-mar}

\newpage

\tableofcontents
\listoffigures
\listoftables

\newpage

\hypertarget{glosario}{%
\chapter*{Glosario}\label{glosario}}
\addcontentsline{toc}{chapter}{Glosario}

faltaría introducir el orden de Lista de Figuras y lista de tablas

\newpage

\hypertarget{resumen}{%
\chapter*{Resumen}\label{resumen}}
\addcontentsline{toc}{chapter}{Resumen}

uufff aquí va el resumen. No olvidar las palabras clave

\newpage

\hypertarget{abstract}{%
\chapter*{Abstract}\label{abstract}}
\addcontentsline{toc}{chapter}{Abstract}

at the end!!! Don't forget the keywords

\mainmatter

\hypertarget{introducciuxf3n-aka.-about}{%
\chapter*{Introducción (aka. About)}\label{introducciuxf3n-aka.-about}}
\addcontentsline{toc}{chapter}{Introducción (aka. About)}

This is a \emph{sample} book written in \textbf{Markdown}\index{markdown}. You can use anything that Pandoc's Markdown supports; for example, a math equation\index{ecuación} \(a^2 + b^2 = c^2\) según Eddelbuettel \& Balamuta (\protect\hyperlink{ref-eddelbuettel2017ExtendingBrief}{2017}).

Este primer capítulo numerado (00-realcap1.Rmd) después de sacarlo del index.Rmd. Se agregan los archivos complementarios del frontmatter que se numerarían en romanos y en el índice ver si aparecen\ldots{}

\hypertarget{usage}{%
\section{Usage}\label{usage}}

Each \textbf{bookdown}\index{bookdown} chapter\index{capítulo} is an .Rmd file, and each .Rmd file can contain one (and only one) chapter. A chapter \emph{must} start with a first-level heading: \texttt{\#\ A\ good\ chapter}, and can contain one (and only one) first-level heading.

Use second-level and higher headings within chapters like: \texttt{\#\#\ A\ short\ section} or \texttt{\#\#\#\ An\ even\ shorter\ section}.

The \texttt{index.Rmd} file is required\index{obligatorio}, and is also your first book chapter. It will be the homepage when you render the book \index{libro}.

\hypertarget{render-book}{%
\section{Render book}\label{render-book}}

You can render the HTML version of this example book without changing anything:

\begin{enumerate}
\def\labelenumi{\arabic{enumi}.}
\item
  Find the \textbf{Build} pane in the RStudio IDE, and
\item
  Click on \textbf{Build Book}, then select your output format, or select ``All formats'' if you'd like to use multiple formats from the same book source files.
\end{enumerate}

Or build the book from the R console:

\begin{Shaded}
\begin{Highlighting}[]
\NormalTok{bookdown}\SpecialCharTok{::}\FunctionTok{render\_book}\NormalTok{()}
\end{Highlighting}
\end{Shaded}

To render this example to PDF \index{PDF} as a \texttt{bookdown::pdf\_book}, you'll need to install XeLaTeX. You are recommended to install TinyTeX \index{TinyTeX}(which includes XeLaTeX): \url{https://yihui.org/tinytex/}.

\hypertarget{preview-book}{%
\section{\texorpdfstring{Preview book \index{vista previa}}{Preview book }}\label{preview-book}}

As you work, you may start a local server to live preview this HTML book. This preview will update as you edit the book when you save individual .Rmd files. You can start the server in a work session by using the RStudio add-in ``Preview book'', or from the R console:

\begin{Shaded}
\begin{Highlighting}[]
\NormalTok{bookdown}\SpecialCharTok{::}\FunctionTok{serve\_book}\NormalTok{()}
\end{Highlighting}
\end{Shaded}

Esta es la prueba en modo visual para insertar \index{insertar}directo una referencia solo con la arroba Wickham (\protect\hyperlink{ref-wickham2019Advanced}{2019}) y los demás (\protect\hyperlink{ref-xie2015}{Xie, 2015}; \protect\hyperlink{ref-wickham2019Advanced}{Wickham, 2019}). En realidad no fununcia; sin embargo, en el modo visual, el atajo Ctrl+SHift+F8 abre el manejador de citas y permite seleccionar la(s) deseada(s) para insertar la bibkey (\protect\hyperlink{ref-R-base}{R Core Team, 2022}).

La primera prueba es con la base generada directo por Zotero \index{Zotero}(Export selected as BibTeX, Saldierna-Calapiz (\protect\hyperlink{ref-saldierna-calapiz2018Efectored}{2018})).

Ahmad \emph{et al.} (\protect\hyperlink{ref-ahmad2016UsingFisher}{2016}) establece que Anscombe (\protect\hyperlink{ref-anscombe1973GraphsStatistical}{1973}) está totalmente equivocado al insertar directo desde Zotero y agregarla al archivo de bibliografía (\protect\hyperlink{ref-bravo2016Teachinghigher}{Bravo \emph{et al.}, 2016}).

Lo que sigue se supone que es \textbf{\emph{Markdown extended}}:

Por ejemplo, la definición de términos

\begin{description}
\item[hábrase visto:]
expresión choyera de asombro!
\item[relación:]
solo para acompletar el punto; ponemos mucho texto para ver como maneja el párrafo. As you work, you may start a local server to live preview this HTML book. This preview will update as you edit the book when you save individual .Rmd files. You can start the server in a work session by using the RStudio add-in ``Preview book'', or from the R console
\end{description}

También hay una lista de pendientes, parecido a la lista ordenada o no

\begin{itemize}
\tightlist
\item[$\square$]
  Write the press release
\item[$\square$]
  Update the website
\item[$\square$]
  Contact the media
\end{itemize}

Así como el resalte de ==lo más importante en el texto== y ver si fununcia. Este marcador extendido para resaltar no es interpretado en \textbf{PDF} (por lo menos).

La referencia la ponemos a pie de página forzada \footnote{\url{https://www.markdownguide.org/cheat-sheet}}. El formato para insertarlo es \^{}{[}texto del pie de página{]}; RMarkdown lo interpreta y lo deja numerado y estandarizado -como aparece en el \emph{Rmd}-.

\hypertarget{hello-bookdown}{%
\chapter*{Hello bookdown}\label{hello-bookdown}}
\addcontentsline{toc}{chapter}{Hello bookdown}

All chapters start with a first-level heading followed by your chapter title, like the line above. There should be only one first-level heading (\texttt{\#}) per .Rmd file.

\hypertarget{a-section}{%
\section{A section}\label{a-section}}

All chapter sections start with a second-level (\texttt{\#\#}) or higher heading followed by your section title, like the sections above and below here. You can have as many as you want within a chapter.

\hypertarget{an-unnumbered-section}{%
\subsection*{An unnumbered section}\label{an-unnumbered-section}}
\addcontentsline{toc}{subsection}{An unnumbered section}

Chapters and sections are numbered by default. To un-number a heading, add a \texttt{\{.unnumbered\}} or the shorter \texttt{\{-\}} at the end of the heading, like in this section.

\hypertarget{r-markdown}{%
\section{R Markdown}\label{r-markdown}}

This is an R Markdown document. Markdown is a simple formatting syntax for authoring HTML, PDF, and MS Word documents. For more details on using R Markdown see \url{http://rmarkdown.rstudio.com}.

When you click the \textbf{Knit} button a document will be generated that includes both content as well as the output of any embedded R code chunks within the document p \(\approx\) 0.07. You can embed an R code chunk like this:

\begin{Shaded}
\begin{Highlighting}[]
\CommentTok{\# summary(cars)}
\FunctionTok{print}\NormalTok{(}\FunctionTok{Pvalue}\NormalTok{(}\FloatTok{0.07}\NormalTok{))}
\end{Highlighting}
\end{Shaded}

\begin{verbatim}
## p $\approx$ 0.07
\end{verbatim}

\hypertarget{cross}{%
\chapter*{Cross-references}\label{cross}}
\addcontentsline{toc}{chapter}{Cross-references}

Cross-references \index{referencia cruzada}make it easier for your readers to find and link to elements in your book.

\hypertarget{chapters-and-sub-chapters}{%
\section{Chapters and sub-chapters}\label{chapters-and-sub-chapters}}

There are two steps to cross-reference any heading:

\begin{enumerate}
\def\labelenumi{\arabic{enumi}.}
\tightlist
\item
  Label the heading: \texttt{\#\ Hello\ world\ \{\#nice-label\}}.

  \begin{itemize}
  \tightlist
  \item
    Leave the label off if you like the automated heading generated based on your heading title: for example, \texttt{\#\ Hello\ world} = \texttt{\#\ Hello\ world\ \{\#hello-world\}}.
  \item
    To label an un-numbered heading, use: \texttt{\#\ Hello\ world\ \{-\#nice-label\}} or \texttt{\{\#\ Hello\ world\ .unnumbered\}}.
  \end{itemize}
\item
  Next, reference the labeled heading anywhere in the text using \texttt{\textbackslash{}@ref(nice-label)}; for example, please see Chapter \ref{cross}.

  \begin{itemize}
  \tightlist
  \item
    If you prefer text as the link instead of a numbered reference use: \protect\hyperlink{cross}{any text you want can go here}.
  \end{itemize}
\end{enumerate}

\hypertarget{captioned-figures-and-tables}{%
\section{Captioned figures and tables}\label{captioned-figures-and-tables}}

Figures \index{figuras}and tables \index{tablas}\emph{with captions} can also be cross-referenced from elsewhere in your book using \texttt{\textbackslash{}@ref(fig:chunk-label)} and \texttt{\textbackslash{}@ref(tab:chunk-label)}, respectively.

See Figure \ref{fig:nice-fig}.

\begin{Shaded}
\begin{Highlighting}[]
\FunctionTok{par}\NormalTok{(}\AttributeTok{mar =} \FunctionTok{c}\NormalTok{(}\DecValTok{4}\NormalTok{, }\DecValTok{4}\NormalTok{, .}\DecValTok{1}\NormalTok{, .}\DecValTok{1}\NormalTok{))}
\FunctionTok{plot}\NormalTok{(pressure, }\AttributeTok{type =} \StringTok{\textquotesingle{}b\textquotesingle{}}\NormalTok{, }\AttributeTok{pch =} \DecValTok{19}\NormalTok{)}
\end{Highlighting}
\end{Shaded}

\begin{figure}

{\centering \includegraphics[width=0.8\linewidth]{figuras/nice-fig-1} 

}

\caption{Here is a nice figure!}\label{fig:nice-fig}
\end{figure}

Don't miss Table \ref{tab:nice-tab}.

\begin{Shaded}
\begin{Highlighting}[]
\NormalTok{knitr}\SpecialCharTok{::}\FunctionTok{kable}\NormalTok{(}
  \FunctionTok{head}\NormalTok{(pressure, }\DecValTok{10}\NormalTok{), }\AttributeTok{caption =} \StringTok{\textquotesingle{}Here is a nice table!\textquotesingle{}}\NormalTok{,}
  \AttributeTok{booktabs =} \ConstantTok{TRUE}
\NormalTok{)}
\end{Highlighting}
\end{Shaded}

\begin{table}

\caption{\label{tab:nice-tab}Here is a nice table!}
\centering
\begin{tabular}[t]{rr}
\toprule
temperature & pressure\\
\midrule
0 & 0.0002\\
20 & 0.0012\\
40 & 0.0060\\
60 & 0.0300\\
80 & 0.0900\\
\addlinespace
100 & 0.2700\\
120 & 0.7500\\
140 & 1.8500\\
160 & 4.2000\\
180 & 8.8000\\
\bottomrule
\end{tabular}
\end{table}

\hypertarget{parts}{%
\chapter*{Parts}\label{parts}}
\addcontentsline{toc}{chapter}{Parts}

You can add parts \index{partes o secciones}to organize one or more book chapters together. Parts can be inserted at the top of an .Rmd file, before the first-level chapter heading in that same file.

Add a numbered part: \texttt{\#\ (PART)\ Act\ one\ \{-\}} (followed by \texttt{\#\ A\ chapter})

Add an unnumbered part: \texttt{\#\ (PART\textbackslash{}*)\ Act\ one\ \{-\}} (followed by \texttt{\#\ A\ chapter})

Add an appendix \index{apéndices}as a special kind of un-numbered part: \texttt{\#\ (APPENDIX)\ Other\ stuff\ \{-\}} (followed by \texttt{\#\ A\ chapter}). Chapters in an appendix are prepended with letters instead of numbers.

\hypertarget{footnotes-and-citations}{%
\chapter*{Footnotes and citations}\label{footnotes-and-citations}}
\addcontentsline{toc}{chapter}{Footnotes and citations}

\hypertarget{footnotes}{%
\section{Footnotes}\label{footnotes}}

Footnotes \index{pie de página}are put inside the square brackets after a caret \texttt{\^{}{[}{]}}. Like this one \footnote{This is a footnote.}.

\hypertarget{citations}{%
\section{Citations}\label{citations}}

Reference items in your bibliography \index{bibliografía}file(s) using \texttt{@key}.

For example, we are using the \textbf{bookdown} package (\protect\hyperlink{ref-R-bookdown}{Xie, 2022}) (check out the last code chunk in index.Rmd to see how this citation key was added) in this sample book, which was built on top of R Markdown and \textbf{knitr} (\protect\hyperlink{ref-xie2015}{Xie, 2015}) (this citation was added manually in an external file book.bib).
Note that the \texttt{.bib} files need to be listed in the index.Rmd with the YAML \texttt{bibliography} key.

The RStudio Visual Markdown Editor \index{modo visual}can also make it easier to insert citations: \url{https://rstudio.github.io/visual-markdown-editing/\#/citations}

\hypertarget{blocks}{%
\chapter*{Blocks}\label{blocks}}
\addcontentsline{toc}{chapter}{Blocks}

\hypertarget{equations}{%
\section{Equations}\label{equations}}

Here is an equation. \index{bloques de contenido}

\begin{equation} 
  f\left(k\right) = \binom{n}{k} p^k\left(1-p\right)^{n-k}
  \label{eq:binom}
\end{equation}

You may refer to using \texttt{\textbackslash{}@ref(eq:binom)}, like see Equation \eqref{eq:binom}.

\hypertarget{theorems-and-proofs}{%
\section{Theorems and proofs}\label{theorems-and-proofs}}

Labeled theorems can be referenced in text using \texttt{\textbackslash{}@ref(thm:tri)}, for example, check out this smart theorem \ref{thm:tri}.

\begin{theorem}
\protect\hypertarget{thm:tri}{}\label{thm:tri}For a right triangle, if \(c\) denotes the \emph{length} of the hypotenuse
and \(a\) and \(b\) denote the lengths of the \textbf{other} two sides, we have
\[a^2 + b^2 = c^2\]
\end{theorem}

Read more here \url{https://bookdown.org/yihui/bookdown/markdown-extensions-by-bookdown.html}.

\hypertarget{callout-blocks}{%
\section{Callout blocks}\label{callout-blocks}}

The R Markdown Cookbook provides more help on how to use custom blocks \index{bloque personal}to design your own callouts: \url{https://bookdown.org/yihui/rmarkdown-cookbook/custom-blocks.html}

\hypertarget{sharing-your-book}{%
\chapter*{Sharing your book}\label{sharing-your-book}}
\addcontentsline{toc}{chapter}{Sharing your book}

\hypertarget{publishing}{%
\section{Publishing}\label{publishing}}

HTML books can be published online \index{libro en línea}, see: \url{https://bookdown.org/yihui/bookdown/publishing.html}

\hypertarget{pages}{%
\section{404 pages}\label{pages}}

By default, users will be directed to a 404 page if they try to access a webpage that cannot be found. If you'd like to customize your 404 page instead of using the default, you may add either a \texttt{\_404.Rmd} or \texttt{\_404.md} file to your project root and use code and/or Markdown syntax.

\hypertarget{metadata-for-sharing}{%
\section{Metadata for sharing}\label{metadata-for-sharing}}

Bookdown HTML books will provide HTML metadata for social sharing on platforms like Twitter, Facebook, and LinkedIn, using information you provide in the \texttt{index.Rmd} YAML. To setup, set the \texttt{url} for your book and the path to your \texttt{cover-image} file. Your book's \texttt{title} and \texttt{description} are also used.

This \texttt{gitbook} \index{formato 'github'}uses the same social sharing data across all chapters in your book- all links shared will look the same.

Specify your book's source repository on GitHub using the \texttt{edit} key under the configuration options in the \texttt{\_output.yml} file, which allows users to suggest an edit by linking to a chapter's source file.

Read more about the features of this output format here:

\url{https://pkgs.rstudio.com/bookdown/reference/gitbook.html}

Or use:

\begin{Shaded}
\begin{Highlighting}[]
\NormalTok{?bookdown}\SpecialCharTok{::}\NormalTok{gitbook}
\end{Highlighting}
\end{Shaded}

\hypertarget{bibliografuxeda}{%
\chapter*{Bibliografía}\label{bibliografuxeda}}
\addcontentsline{toc}{chapter}{Bibliografía}

Ohhh ma, llegamos a las referencias y aparecen en el índice y bien cabeceadas en su propio capítulo
\newline

\hypertarget{refs}{}
\begin{CSLReferences}{1}{0}
\leavevmode\vadjust pre{\hypertarget{ref-ahmad2016UsingFisher}{}}%
Ahmad, N. et al. 2016. Using Fisher information to track stability in multivariate systems. \emph{Royal Society Open Science}, 3: 160582. ISSN: 2054-5703, 2054-5703 Available at: \url{https://royalsocietypublishing.org/doi/10.1098/rsos.160582}.

\leavevmode\vadjust pre{\hypertarget{ref-anscombe1973GraphsStatistical}{}}%
Anscombe, F.J. 1973. Graphs in Statistical Analysis. \emph{The American Statistician}, 27: 17--21. ISSN: 0003-1305 Available at: \url{https://www.tandfonline.com/doi/abs/10.1080/00031305.1973.10478966}.

\leavevmode\vadjust pre{\hypertarget{ref-bravo2016Teachinghigher}{}}%
Bravo, A. et al. 2016. Teaching for higher levels of thinking: developing quantitative and analytical skills in environmental science courses. \emph{Ecosphere}, 7: e01290. ISSN: 2150-8925 Available at: \url{https://esajournals.onlinelibrary.wiley.com/doi/abs/10.1002/ecs2.1290}.

\leavevmode\vadjust pre{\hypertarget{ref-eddelbuettel2017ExtendingBrief}{}}%
Eddelbuettel, D. \& J.J. Balamuta. 2017. \emph{Extending R with C++: A Brief Introduction to Rcpp}. PeerJ Preprints. Available at: \url{https://peerj.com/preprints/3188}.

\leavevmode\vadjust pre{\hypertarget{ref-R-base}{}}%
R Core Team. 2022. \emph{R: A language and environment for statistical computing}. R Foundation for Statistical Computing, Vienna, Austria. Available at: \url{https://www.R-project.org/}.

\leavevmode\vadjust pre{\hypertarget{ref-saldierna-calapiz2018Efectored}{}}%
Saldierna-Calapiz, D. 2018. \emph{Efecto de la red de zonas de refigio pesquero San Cosme a Punta Coyote, B.C.S., México, en la comunidad de peces}. Tesis de Maestría. Centro Interdisciplinario de Ciencias Marinas del IPN, La Paz, Baja California Sur. 98 pp.

\leavevmode\vadjust pre{\hypertarget{ref-wickham2019Advanced}{}}%
Wickham, H. 2019. \emph{Advanced R}. 2nd Edition. Chapman \& Hall/CRC, Boca Raton, FL. 604 pp. ISBN: 978-0-8153-8457-1

\leavevmode\vadjust pre{\hypertarget{ref-R-bookdown}{}}%
Xie, Y. 2022. \emph{Bookdown: Authoring books and technical documents with r markdown}. Available at: \url{https://CRAN.R-project.org/package=bookdown}.

\leavevmode\vadjust pre{\hypertarget{ref-xie2015}{}}%
Xie, Y. 2015. \emph{Dynamic documents with {R} and knitr}. 2nd ed. Chapman; Hall/CRC, Boca Raton, Florida. Available at: \url{http://yihui.org/knitr/}.

\end{CSLReferences}

\printindex

\end{document}
